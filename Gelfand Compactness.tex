\documentclass{amsart}

\usepackage{lipsum}
\usepackage{amsfonts}
\usepackage{graphicx}
\usepackage{amsmath}
\usepackage{amssymb}
\usepackage[english]{babel}
\usepackage{tikz-cd}
\usepackage{hyperref}
\usepackage{enumerate}


\newtheorem{theorem}{Theorem}[section]
\newtheorem{lemma}{Lemma}[section]
\newtheorem{proposition}{Proposition}[section]
\newtheorem{corollary}{Corollary}[section]
\newtheorem{definition}{Definition}[section]
\newtheorem{observation}{Observation}[section]
\newtheorem{example}{Example}[section]

\newenvironment{AMS}{****}{****}

\begin{document}

\title{Gelfand Compactness}

\author{Frank Murphy-Hernandez}
\address{Facultad de Ciencias, UNAM, Mexico City}
\email{murphy@ciencias.unam.mx}

\subjclass[2000]{Primary ****, ****; Secondary ****, ****}

\date{\today}

\keywords{,,}

\begin{abstract}

\end{abstract}

\maketitle

%%%%%%%%%%%%%%%%%%%%%%%%%%%%%%%%%%%%%%%%%%%%%%%%%%%%%%%%%%%%%%%%%%%%
\section*{Introduction}

In this paper the rings are associative but we do not ask them to have unit.



%%%%%%%%%%%%%%%%%%%%%%%%%%%%%%%%%%%%%%%%%%%%%%%%%%%%%%%%%%%%%%%%%%%%
\section{Preliminaries}

\cite{wisbauer2018foundations}


%%%%%%%%%%%%%%%%%%%%%%%%%%%%%%%%%%%%%%%%%%%%%%%%%%%%%%%%%%%%%%%%%%%%
\section{Rings with local units}

\begin{definition}
Let $R$ be a ring. We say that $R$ is a ring with local units, if for any finitely many $x_1,\dots,x_n\in R$ there is an idempotent $e\in R$ with $a_1,\dots,a_n\in eRe$.
\end{definition}

\begin{definition}
Let $R$ be a ring. We say that $R$ has enough idempotents, if there exists a family $\{e_\lambda\}_{\lambda\in\Lambda}$ of pairwise orthogonal idempotents of elements in $R$ with $R=\bigoplus_{\lambda\in\Lambda}Re_\lambda=\bigoplus_{\lambda\in\Lambda}e_\lambda R$. In this case $\{e_\lambda\}_{\lambda\in\Lambda}$ is called a complete family of idempotents in $R$. 
\end{definition}

\begin{definition}
Let $R$ be a ring. We say that $R$ is a $s$-unital ring, if for any finitely many $x_1,\dots,x_n\in R$ there is $y\in R$ with $x_iy=x_i=yx_i$ for $i=1,\dots,n$.
\end{definition}

\begin{definition}
Let $R$ be a ring. We say that $R$ is a firm ring, if the canonical morphism $\mu\colon R\otimes_R R\longrightarrow R$ given by $\mu(r\otimes s)=rs$ is an isomorphism.
\end{definition}

\begin{definition}
Let $R$ be a ring. We say that $R$ is idempotent, if $R^2=R$, that is, that for any $x\in R$ there are $x_1,\dots,x_n,y_1,\dots,y_n\in R$ such that $x=\sum_{i=1}^nx_iy_i$.
\end{definition}


%%%%%%%%%%%%%%%%%%%%%%%%%%%%%%%%%%%%%%%%%%%%%%%%%%%%%%%%%%%%%%%%%%%%
\section{Paracompact space}

\cite{dieudonne1944generalisation}





%%%%%%%%%%%%%%%%%%%%%%%%%%%%%%%%%%%%%%%%%%%%%%%%%%%%%%%%%%%%%%%%%%%%
\section{Metacompact space}







%%%%%%%%%%%%%%%%%%%%%%%%%%%%%%%%%%%%%%%%%%%%%%%%%%%%%%%%%%%%%%%%%%%%
\section{Orthocompact space}







%%%%%%%%%%%%%%%%%%%%%%%%%%%%%%%%%%%%%%%%%%%%%%%%%%%%%%%%%%
\bibliography{biblio}
\bibliographystyle{plain}



\end{document}

%------------------------------------------------------------------------------
% End of journal.tex
%------------------------------------------------------------------------------
